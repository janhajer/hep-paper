% \iffalse meta-comment
%
% Copyright (C) 2019-2023 by Jan Hajer
% -----------------------------------
%
% This file may be distributed and/or modified under the
% conditions of the LaTeX Project Public License, either version 1.3c
% of this license or (at your option) any later version.
% The latest version of this license is in:
%
% http://www.latex-project.org/lppl.txt
%
% and version 1.3c or later is part of all distributions of LaTeX
% version 2005/12/01 or later.
%
% \fi
%
% \iffalse

%<package>\NeedsTeXFormat{LaTeX2e}[2005/12/01]
%<package>\ProvidesPackage{hep-math}[2023/07/01 v1.2 Math and physics macros]
%<documentation>\ProvidesFile{hep-math-documentation.tex}[2023/07/01 v1.2 hep-math documentation]
%
%<*documentation>
\RequirePackage[l2tabu, orthodox]{nag}
\documentclass{ltxdoc}

\renewcommand\theCodelineNo{\rmfamily\tstyle\footnotesize\arabic{CodelineNo}}
\AtBeginEnvironment{macrocode}{\renewcommand{\ttdefault}{clmt}}
\renewcommand{\MacroFont}{\codestyle}
\AtBeginDocument{\DeleteShortVerb{\|}}
\AtBeginDocument{\MakeShortVerb{\"}}
\EnableCrossrefs
\CodelineIndex
\RecordChanges

\usepackage{hologo}

\usepackage[parskip,oldstyle,font=10pt]{hep-paper}
\bibliography{bibliography}

\setlength{\fboxsep}{1pt}
\newenvironment{columns}[1][.5]{%
  \par\vspace{-\bigskipamount}%
  \begin{minipage}[t]{\linewidth}%
  \begin{minipage}[t]{#1\linewidth}%
  \def\column{%
    \end{minipage}%
    \begin{minipage}[t]{\linewidth-#1\linewidth}%
  }%
}{\end{minipage}\end{minipage}\par}
%</documentation>

%<*driver>
\expandafter\newif\csname ifshort\endcsname
\shortfalse
\begin{document}
\DocInput{hep-math-implementation.dtx}
\end{document}
%</driver>
%
% \fi
%
% \CheckSum{712}
%
% \CharacterTable
%  {Upper-case    \A\B\C\D\E\F\G\H\I\J\K\L\M\N\O\P\Q\R\S\T\U\V\W\X\Y\Z
%   Lower-case    \a\b\c\d\e\f\g\h\i\j\k\l\m\n\o\p\q\r\s\t\u\v\w\x\y\z
%   Digits        \0\1\2\3\4\5\6\7\8\9
%   Exclamation   \!     Double quote  \"     Hash (number) \#
%   Dollar        \$     Percent       \%     Ampersand     \&
%   Acute accent  \'     Left paren    \(     Right paren   \)
%   Asterisk      \*     Plus          \+     Comma         \,
%   Minus         \-     Point         \.     Solidus       \/
%   Colon         \:     Semicolon     \;     Less than     \<
%   Equals        \=     Greater than  \>     Question mark \?
%   Commercial at \@     Left bracket  \[     Backslash     \\
%   Right bracket \]     Circumflex    \^     Underscore    \_
%   Grave accent  \`     Left brace    \{     Vertical bar  \|
%   Right brace   \}     Tilde         \~}
%
% \changes{v1.0}{2021/09/01}{Initial version of the style file.}
% \changes{v1.1}{2023/07/01}{Bug fixes.}
%
% \ifshort
%<*documentation>
% \fi
%
\GetFileInfo{hep-math.sty}

\title{The \software{hep-math} package\thanks{This document corresponds to \software{hep-math}~\fileversion.}}
\subtitle{Extended math macros}
\author{Jan Hajer \email{jan.hajer@tecnico.ulisboa.pt}}
\date{\filedate}

% \ifshort
\begin{document}
% \fi

\newgeometry{vscale=.8, vmarginratio=3:4, includeheadfoot, left=11em, marginparwidth=4.6cm, marginparsep=3mm, right=7em}

\maketitle

\begin{abstract}
The \software{hep-math} package provides some additional features beyond the \software{mathtools} and \software{amsmath} packages.
\end{abstract}

To use the package place "\usepackage{hep-math}" in the preamble.

The \software{mathtools} \cite{mathtools} package is loaded, which in turn loads the \hologo{AmSLaTeX} \software{amsmath} \cite{amsmath} package.
Horizontal spacing in inline equations and page breaks in block equations are marginally adjusted.
\DescribeMacro{\left}
\DescribeMacro{\right}
Spacing around "\left" and "\right" is fixed with the \software{mleftright} package \cite{mleftright}.

\section{Macros}

\DescribeMacro{\mathdef}
The "\mathdef"\marg{name}\oarg{arguments}\marg{code} macro \prefix{re}{defines} macros only within math mode without changing the text mode definition.

\DescribeMacro{\i}
\DescribeMacro{\d}
The imaginary unit "\i" and the differential "\d" are defined using this functionality.

\DescribeMacro{\overline}
The "\overline" macro is adjusted to \overline{work also outside} of math mode using the \software{soul} \cite{soul} package.

\DescribeMacro{\oset}
\DescribeMacro{\overleft}
\DescribeMacro{\overright}
\DescribeMacro{\overleftright}
A better looking over left right arrow is defined \ie $\overleftright{\partial}$ using a new "\oset"\marg{over}\marg{math} functionality.

\DescribeMacro{\diag}
\DescribeMacro{\sgn}
Diagonal matrix "\diag", signum "\sgn", trace "\tr", "\Tr", and "\rank" operators are defined.

\DescribeMacro{\Re}
\DescribeMacro{\Im}
The real and imaginary projectors are redefined to look like ordinary operators.

\DescribeMacro{\sin}
\DescribeMacro{\cos}
\DescribeMacro{\tan}
"\cos" and "\tan" are adjusted to have the same height as "\sin".

\DescribeMacro{\accsc}
"\arccsc" and other inverse trigonometric functions are defined.

\subsection{Fractions and units}

\DescribeMacro{\unit}
\DescribeMacro{\inv}
The correct spacing for units is provided by the macro "\unit"\oarg{value}\marg{unit} from the \software{units} package \cite{units} which can also be used in text mode.
The macro "\inv"\oarg{power}\marg{text} allows to avoid math mode also for inverse units such as \unit[5]{\inv{fb}} typeset via "\unit[5]{\inv{fb}}".

\DescribeMacro{\nicefrac}
\DescribeMacro{\flatfrac}
\DescribeMacro{\textfrac}
The "\frac"\marg{number}\marg{number} macro is accompanied by "\nicefrac"\linebreak[1]\marg{number}\linebreak[1]\marg{number}, "\textfrac"\marg{number}\marg{number}, and "\flatfrac"\marg{number}\marg{number} leading to $\frac12$, $\nicefrac12$, \textfrac12, and $\flatfrac12$.
The "\textfrac" macro is mostly intended if a font with oldstyle numerals is used.

Some macros of the \software{physics} package \cite{physics} are reimplemented with a more conventional typesetting in mind.
Finer details about mathematical typesetting can be found in \cite{gregorio:2020}.

\subsection{Differentials and derivatives}

\DescribeMacro{\differential}
\DescribeMacro{\newderivative}
\DescribeMacro{\newpartialderivative}
The three macros "\differential"\marg{symbol}, "\newderivative" "{\name}"\marg{symbol}, and "\newpartialderivative" "{\name}"\marg{symbol} allow to define a differential with correct spacing, a derivative using this differential, and if necessary a partial derivative that can handle three dimensional derivatives.

\DescribeMacro{\d}
\DescribeMacro{\dv}
These macros are used for the usual differential and derivative, producing $\d x$ via "\d x" and
\begin{center}
\begin{tabular}{*4c}
"\dv[f]x" & "\dv*[f]x^n" & "\dv[f]x*^n" & "\dv*[f]x*^n" \\
$\dv[f]x$ & $\dv*[f]x^n$ & $\dv[f]x*^n$ & $\dv*[f]x*^n$ \\
"\dv xf" & "\dv*xf" & "\dv x*f" & "\dv*x*f" \\
$\dv xf$ & $\dv*xf$ & $\dv x*f$ & $\dv*x*f$
\end{tabular}
\end{center}
via "\dv*"\oarg{f}\marg{x}"*""^"\marg{n}.
Upright differential can be produced via "\renewcommand""{\diffsymbol}" "{\mathrm d}".
The differential takes care of the correct spacing as long as it is placed at the end of the integral $\int f(x)\d x$.
In order to archive correct spacing when it is placed at the beginning of the integral it is advisable to place the whole expression in a "\mathop{\int\d x} f(x)"  such that $\mathop{\int\d x} f(x)$.

\DescribeMacro{\pd}
\DescribeMacro{\pdv}
Similarly a partial differential and derivative are defined that can be used according to "\pdv*"\oarg{f}\marg{x}"*""^"\marg{a}\oarg{y}"^"\marg{b}\oarg{z}"^"\marg{c}.
\begin{center}
\begin{tabular}{*4c}
"\pdv[f]x" & "\pdv[f]x[y]" & "\pdv[f]x^3" & "\pdv[f]x^2[y]" \\
$\pdv[f]x$ & $\pdv[f]x[y]$ & $\pdv[f]x^3$ & $\pdv[f]x^2[y]$ \\
"\pdv[f]x^2[y]^3" & "\pdv[f]x[y]^3" & "\pdv x[y]f"\\
$\pdv[f]x^2[y]^3$ & $\pdv[f]x[y]^3$ & $\pdv x[y]f$
\end{tabular}
\end{center}

\DescribeMacro{\var}
\DescribeMacro{\fdv}
Similarly a functional variation and functional derivative are defined.

\DescribeMacro{\cancel}
\DescribeMacro{\slashed}
The "\cancel"\marg{characters} macro from the \software{cancel} package \cite{cancel} and the "\slashed" \marg{character} macro from the \software{slashed} package \cite{slashed} allow to $\cancel{\text{cancel}}$ math and use the Dirac slash notation \ie $\slashed \pd$, respectively.

\subsection{Paired delimiters}

\DescribeMacro{\abs}
\DescribeMacro{\norm}
\begin{center}
\begin{tabular}{*8r}
"\abs x" & "\norm x" & "\norm[2]x" & "\norm*[2]x" \\
$\abs x$ & $\norm x$ & $\norm[2]x$ & $\norm*[2]x$
\end{tabular}
\end{center}

\DescribeMacro{\eval}
\DescribeMacro{\order}
\begin{center}
\begin{tabular}{*8r}
"\order x" & "\eval x_0^\infty" & "\eval* x_0^\infty" \\
$\order x$ & $\eval x_0^\infty$ & $\eval* x_0^\infty$
\end{tabular}
\end{center}

\DescribeMacro{\newpair}
\DescribeMacro{\comm}
\DescribeMacro{\acomm}
The "\newpair"\marg{name}\marg{left delim} \marg{right delim}"_"\marg{subscript}"^"\marg{superscript} macro is defined and used for the definition of \prefix{anti}{commutators} and Poisson brackets.
\begin{center}
\begin{tabular}{*8r}
"\pb xy" & "\comm xy" & "\acomm xy" \\
$\pb xy$ & $\comm xy$ & $\acomm xy$
\end{tabular}
\end{center}
They can easily be redefined using \eg "\newpair\comm\lbrack\rbrack_-".

\DescribeMacro{\bra}
\DescribeMacro{\ket}
\DescribeMacro{\braket}
\DescribeMacro{\ketbra}
\DescribeMacro{\mel}
\DescribeMacro{\ev}
\DescribeMacro{\vev}
Macros for the bra-ket notation are introduced.
\begin{center}
\begin{tabular}{*4c}
"\bra x" & "\ket x" & "\braket xy" & "\ketbra xy" \\
$\bra x$ & $\ket x$ & $\braket xy$ & $\ketbra xy$ \\
"\mel xyz" & "\ev x" & "\ev[\Omega] x" & "\vev x" \\
$\mel xyz$ & $\ev x$ & $\ev[\Omega] x$ & $\vev x$
\end{tabular}
\end{center}

\DescribeMacro{\column}
\DescribeMacro{\row}
Macros for row and column vectors are introduced together with a symbol for transpose vectors.
\begin{center}
\begin{tabular}{*4c}
"\column{x,y,z}" & "\row{x,y,z}^\trans" \\
$\column{x,y,z}$ & $\row{x,y,z}^\trans$ \\
\end{tabular}
\end{center}

\section{Environments}

\DescribeMacro{eqnarray}
The "eqnarray" environment is depreciated, the "split", "multline", "align", "multlined", "aligned", "alignedat", and "cases" environments of the \software{amsmath} and \software{mathtools} packages should be used instead.

\DescribeMacro{equation}
Use the "equation" environment for short equations.
\begin{columns}
\begin{verbatim}
\begin{equation}
  left = right \ .
\end{equation}
\end{verbatim}
\column
\begin{equation}
\framebox[2em]{left\strut} = \framebox[7em]{right\strut} \ .
\end{equation}
\end{columns}

\DescribeMacro{multline}
Use the "multline" environment for longer equations.
\begin{columns}
\begin{verbatim}
\begin{multline}
  left = right 1 \\
  + right 2 \ .
\end{multline}
\end{verbatim}
\column
\begin{multline}
\framebox[2em]{left\strut} = \framebox[7em]{right 1\strut} \\
\framebox[7em]{+ right 2\strut} \ .
\end{multline}
\end{columns}

\DescribeMacro{split}
Use the "split" sub environment for equations in which multiple equal signs should be aligned.
\begin{columns}
\begin{verbatim}
\begin{equation} \begin{split}
  left &= right 1 \\
  &= right 2 \ .
\end{split} \end{equation}
\end{verbatim}
\column
\begin{equation}
\begin{split}
\framebox[2em]{left\strut} &= \framebox[7em]{right 1\strut} \\
&= \framebox[7em]{right 2\strut} \ .
\end{split}
\end{equation}
\end{columns}

\DescribeMacro{align}
Use the "align" environment for the vertical alignment and horizontal distribution of multiple equations.
\begin{columns}
\begin{verbatim}
\begin{subequations} \begin{align}
  left &= right \ , &
  left &= right \ , \\
  left &= right \ , &
  left &= right \ .
\end{align} \end{subequations}
\end{verbatim}
\column
\begin{subequations}
\begin{align}
\framebox[2em]{left\strut} &= \framebox[3em]{right\strut} \ , &
\framebox[2em]{left\strut} &= \framebox[3em]{right\strut} \ , \\
\framebox[2em]{left\strut} &= \framebox[3em]{right\strut} \ , &
\framebox[2em]{left\strut} &= \framebox[3em]{right\strut} \ .
\end{align}
\end{subequations}
\end{columns}

\DescribeMacro{aligned}
Use the "aligned" environment within a "equation" environment if the aligned equations should be labeled with a single equation number.

\DescribeMacro{multlined}
Use the "multlined" environment if either "split" or "align" contain very long lines.
\begin{columns}
\begin{verbatim}
\begin{equation} \begin{split}
  left &= right 1 \\ &=
  \begin{multlined}[t]
     right 2 \\ + right 3 \ .
  \end{multlined}
\end{split} \end{equation}
\end{verbatim}
\column
\begin{equation}
\begin{split}
\framebox[2em]{left\strut} &= \framebox[7em]{right 1\strut} \\ &=
 \begin{multlined}[t]
  \framebox[7em]{right 2\strut} \\
  \framebox[7em]{+ right 3\strut} \ .
\end{multlined}
\end{split}
\end{equation}
\end{columns}

\DescribeMacro{alignat}
Use the "alignat" environment together with the "\mathllap" macro for the alignment of multiple equations with vastly different lengths.
\begin{columns}
\begin{verbatim}
\begin{subequations}
\begin{alignat}{2}
  left &= long right && \ , \\
  le. 2 &= ri. 2 \ , &
  \mathllap{le. 3 = ri. 3} & \ .
\end{alignat}
\end{subequations}
\end{verbatim}
\column
\begin{subequations}
\begin{alignat}{2}
\framebox[2em]{left\strut} &=
\framebox[11em]{long right\strut} && \ , \\
\framebox[2em]{le.\ 2\strut}
&= \framebox[2.5em]{ri.\ 2\strut} \ , &
\mathllap{\framebox[2em]{le.\ 3\strut}
= \framebox[2.5em]{ri.\ 3\strut}} & \ .
\end{alignat}
\end{subequations}
\end{columns}

As a rule of thumb if you have to use "\notag", "\nonumber", or perform manual spacing via "\quad" you are probably using the wrong environment.

% \ifshort
\printbibliography

\end{document}
%
%</documentation>
% \fi
%
% \StopEventually{
% \printbibliography
% \PrintChanges
% }
%
% \appendix
%
% \section{Implementation}
%
%<*package>
%
% Load the \software{mathtools} package \cite{mathtools} which loads the \software{amsmath} package \cite{amsmath}.
% Allow page breaks within equations if necessary.
% Adjust the thick and med mu skips slightly.
%    \begin{macrocode}
\RequirePackage{mathtools}
\mathtoolsset{centercolon}
\allowdisplaybreaks[1]
\thickmuskip=5mu plus 3mu minus 1mu
\medmuskip=4mu plus 2mu minus 3mu
%    \end{macrocode}
%
% \begin{macro}{\mathdef}
% Define the "\mathdef"\marg{name}\oarg{arguments}\marg{macro} macro which \prefix{re}{defines} macros in math mode only.
% This macro is implemented using the \software{xparse} package \cite{xparse}.
%    \begin{macrocode}
\RequirePackage{xparse}
\DeclareDocumentCommand{\mathdef}{mO{0}om}{%
  \expandafter\let\csname hep@text\string#1\endcsname=#1
  \expandafter\newcommand\csname hep@math\string#1%
  \IfNoValueTF{#3}{\endcsname[#2]}{\endcsname[#2][#3]}{#4}
  \DeclareRobustCommand#1{%
    \ifmmode
      \expandafter\let\expandafter\next\csname%
      hep@math\string#1\endcsname%
    \else
      \expandafter\let\expandafter\next\csname%
      hep@text\string#1\endcsname%
    \fi
    \next
  }%
}
%    \end{macrocode}
% \end{macro}
%
% \begin{macro}{\i}
% Provide an upright imaginary unit in math mode.
%    \begin{macrocode}
\newcommand{\imaginaryunit}{\text{i}}
\AtBeginDocument{\mathdef{\i}{\imaginaryunit}}
%    \end{macrocode}
% \end{macro}
%
% \begin{macro}{\overline}
% Redefine "\overline" to be a text macro using the \software{soul} package \cite{soul}.
% Extend it as a math macro with the original definition from the \software{amsmath} package \cite{amsmath}.
%    \begin{macrocode}
\RequirePackage{soul}
% \def\overline#1{{\renewcommand{\ULdepth}{-1.9ex}{}\uline{#1}}}
\newcommand\textoverline[1]{{\setul{-1.9ex}{}\ul{#1}}}
\let\overline\textoverline
\DeclareRobustCommand{\over@line}[1]{\@@overline{#1}}
\mathdef{\overline}{\over@line}
\newcommand\hep@widebar[1]{%
  \mkern2.5mu\overline{\mkern-2.5mu#1\mkern-.5mu}\mkern.5mu%
}
\newcommand\widebar[1]{%
  \settowidth{\dimen0}{\ensuremath{#1}}%
  \ifdim\dimen0>.475em\hep@widebar{#1}\else\bar{#1}\fi%
}
%    \end{macrocode}
% \end{macro}
%
% \begin{macro}{\oset}
% Define a new overset macro "\oset"\oarg{offset}\marg{over}\marg{base}
%    \begin{macrocode}
\newcommand{\oset}[3][-1pt]{%
  \text{\raisebox{.2ex}{$\mathop{#3}\limits^{%
    \vbox to#1{\kern-2\ex@\hbox{$\scriptscriptstyle#2$}\vss}%
  }$}}%
}
%    \end{macrocode}
% \end{macro}
% \begin{macro}{\overleftright}
% Define a over left right arrow "\overleftright"\marg{base}.
%    \begin{macrocode}
\newcommand{\overleft}[1]{\oset{\leftarrow}{#1}}
\newcommand{\overright}[1]{\oset{\rightarrow}{#1}}
\newcommand{\overleftright}[1]{\oset{\leftrightarrow}{#1}}
%    \end{macrocode}
% \end{macro}
%
% \begin{macro}{eqnarray}
% Undefine the "eqnarray" environment if not prevented by package option.
%    \begin{macrocode}
% \newif\ifhep@eqnarray\hep@eqnarraytrue
% \ifhep@eqnarray\else
%   \let\eqnarray\@undefined
%   \let\endeqnarray\@undefined
% \fi
%    \end{macrocode}
% \end{macro}
%
% \subsection{Operators}
%
% \begin{macro}{\tr}
% \begin{macro}{\Tr}
% \begin{macro}{\rank}
% \begin{macro}{\erf}
% \begin{macro}{\Res}
% \begin{macro}{\sgn}
% \begin{macro}{\sgn}
% \begin{macro}{\diag}
% Provide the "\diag", "\sgn", and some other operators.
%    \begin{macrocode}
\DeclareMathOperator{\tr}{tr}
\DeclareMathOperator{\Tr}{Tr}
\DeclareMathOperator{\rank}{rank}
\DeclareMathOperator{\erf}{erf}
\DeclareMathOperator{\Res}{Res}
\DeclareMathOperator{\sgn}{sgn}
\DeclareMathOperator{\diag}{diag}
\let\det\relax\DeclareMathOperator{\det}{det}
%    \end{macrocode}
% \end{macro}
% \end{macro}
% \end{macro}
% \end{macro}
% \end{macro}
% \end{macro}
% \end{macro}
% \end{macro}

% \begin{macro}{\Re}
% \begin{macro}{\Im}
% Redefine the real and imaginary projectors.
%    \begin{macrocode}
\let\Re\relax\DeclareMathOperator{\Re}{Re}
\let\Im\relax\DeclareMathOperator{\Im}{Im}
%    \end{macrocode}
% \end{macro}
% \end{macro}

% \begin{macro}{\transpose}
% \begin{macro}{\trans}
% Define a transpose symbol.
%    \begin{macrocode}
\RequirePackage{amssymb}
\newcommand*{\hep@transpose}[2]{\raisebox{\depth}{$\m@th#1\intercal$}}
\newcommand*{\transpose}{{\mathpalette\hep@transpose{}}}
\let\trans\transpose
%    \end{macrocode}
% \end{macro}
% \end{macro}



% \subsubsection{Trigonometric functions}
%
% \begin{macro}{\cos}
% \begin{macro}{\tan}
% Adjust the height of of cos and tan to be equal to sin.
%    \begin{macrocode}
\let\cos\undefined\DeclareMathOperator{\cos}{cos\vphantom{i}}
\let\tan\undefined\DeclareMathOperator{\tan}{tan\vphantom{i}}
%    \end{macrocode}
% \end{macro}
% \end{macro}
% \begin{macro}{\arccsc}
% \begin{macro}{\arcsec}
% \begin{macro}{\arccot}
% Define arc operators.
%    \begin{macrocode}
\DeclareMathOperator{\arccsc}{arccsc}
\DeclareMathOperator{\arcsec}{arcsec}
\DeclareMathOperator{\arccot}{arccot}
%    \end{macrocode}
% \end{macro}
% \end{macro}
% \end{macro}
% \begin{macro}{\asin}
% \begin{macro}{\acos}
% \begin{macro}{\atan}
% \begin{macro}{\acsc}
% \begin{macro}{\asec}
% \begin{macro}{\acot}
% Define shorthand for arc operators.
%    \begin{macrocode}
\DeclareMathOperator{\asin}{asin}
\DeclareMathOperator{\acos}{acos}
\DeclareMathOperator{\atan}{atan}
\DeclareMathOperator{\acsc}{acsc}
\DeclareMathOperator{\asec}{asec}
\DeclareMathOperator{\acot}{acot}
%    \end{macrocode}
% \end{macro}
% \end{macro}
% \end{macro}
% \end{macro}
% \end{macro}
% \end{macro}
% \begin{macro}{\csch}
% \begin{macro}{\sech}
% Define csch and sech operators.
%    \begin{macrocode}
\DeclareMathOperator{\csch}{csch}
\DeclareMathOperator{\sech}{sech}
%    \end{macrocode}
% \end{macro}
% \end{macro}
%
% \subsection{Units and fractions}
%
% \begin{macro}{\unit}
% Load the \software{units} package \cite{units} which provides the "\units" and "\nicefrac" macros.
% Patch the "\unit" macros to behave like "\mathinner" within an equation
%    \begin{macrocode}
\RequirePackage{units}
\let\oldunit\unit
\renewcommand{\unit}[2][]{%
  \ifthenelse{\boolean{mmode}}{%
    \mathinner{\oldunit[#1]{#2}}%
  }{%
     \oldunit[#1]{#2}%
  }%
}
\let\oldunitfrac\unitfrac
\renewcommand{\unitfrac}[3][]{%
  \ifthenelse{\boolean{mmode}}{%
    \mathinner{\oldunitfrac[#1]{#2}{#3}}%
  }{%
    \oldunitfrac[#1]{#2}{#3}%
  }%
}
%    \end{macrocode}
% \end{macro}
%
% \begin{macro}{\inv}
% Provide a macro for the inverse, useful in combination with the unit macro in text mode.
%    \begin{macrocode}
\newcommand{\inv}[2][1]{#2\ensuremath{^{-#1}}}
%    \end{macrocode}
% \end{macro}
%
% \begin{macro}{\textfrac}
% Provide the "\textfrac" macro useful in combination with a font using lining numerals.
%    \begin{macrocode}
\newcommand{\textfrac}[2]{\ensuremath{\nicefrac{\text{#1}}{\text{#2}}}}
%    \end{macrocode}
% \end{macro}
%
% \begin{macro}{\flatfrac}
% Provide a flat fraction.
%    \begin{macrocode}
\DeclarePairedDelimiterX{\hep@flatfrac}[2]{.}{.}{%
  \kern-\nulldelimiterspace#1\delimsize/%
  \hep@left@delim#2\kern-\nulldelimiterspace%
}
\NewDocumentCommand{\flatfrac}{somm}{%
  \mathinner{%
    \IfBooleanTF{#1}{%
      \hep@flatfrac*{#3}{#4}%
    }{%
      \IfNoValueTF{#2}{\hep@left@delim#3/\hep@left@delim#4%
      }{%
       \hep@flatfrac[#2]{#3}{#4}%
      }%
    }%
  }%
}
%    \end{macrocode}
% \end{macro}
%
% \subsubsection{Differentials and derivatives}
%
% \begin{macro}{\int}
% Redefine "\int".
%    \begin{macrocode}
% \let\hep@int\int
% \RenewDocumentCommand{\int}{oe{_^}}{%
%   \def\temp{\hep@int_{\IfValueT{#2}{#2}}^{\IfValueT{#3}{#3}}}%
%   \IfValueTF{#1}{\mathop{\temp#1}}{\temp}%
% }
%    \end{macrocode}
% \end{macro}
%
% \begin{macro}{\differential}
% Define a generic differential "\differential".
%    \begin{macrocode}
\newcommand{\differential}[1]{\mathop{}\!#1}
%    \end{macrocode}
% \end{macro}
%
% \begin{macro}{\newderivative}
% Define a generic derivative.
%    \begin{macrocode}
\newcommand\newderivative[2]{
  \NewDocumentCommand{#1}{somse{^}}{%
    \IfBooleanTF{##4}{%
      \IfBooleanTF{##1}{\nicefrac}{\frac}%
    }{%
      \IfBooleanTF{##1}{\flatfrac}{\dfrac}%
    }{%
      \differential#2\IfValueT{##5}{^{##5\!}}\IfValueT{##2}{##2}%
    }{%
      \differential#2{{}##3}\IfValueT{##5}{^{##5}}%
    }%
  }
}
%    \end{macrocode}
% \end{macro}
%
% \begin{macro}{\newpartialderivative}
% Define a generic partial derivative
%    \begin{macrocode}
\newcommand\newpartialderivative[2]{
  \NewDocumentCommand{#1}{somsE{^}{1}oE{^}{1}oE{^}{1}}{%
    \def\hep@one{\IfValueTF{##6}{##7}{0}}
    \def\hep@two{\IfValueTF{##8}{##9}{0}}
    \def\hep@sum{\the\numexpr##5+\hep@one+\hep@two\relax}
    \IfBooleanTF{##4}{%
      \IfBooleanTF{##1}{\nicefrac}{\frac}%
    }{%
      \IfBooleanTF{##1}{\flatfrac}{\dfrac}%
    }{%
      \differential#2\ifnum\hep@sum=1\relax\else{^{\hep@sum\!}}\fi
      \IfValueT{##2}{##2}%
    }{%
      \differential#2{{}##3}\ifnum##5=1\relax\else{^{##5}}\fi%
      \IfValueT{##6}{#2##6\ifnum##7=1\relax\else{^{##7}}\fi}%
      \IfValueT{##8}{#2##8\ifnum##9=1\relax\else{^{##9}}\fi}%
    }%
  }
}
%    \end{macrocode}
% \end{macro}
%
% \begin{macro}{\diffsymbol}
% \begin{macro}{\diff}
% \begin{macro}{\d}
% \begin{macro}{\derivative}
% \begin{macro}{\dv}
% Define the differential "\d" and the usual derivative.
%    \begin{macrocode}
\providecommand{\diffsymbol}{d}
\newcommand{\diff}{\differential\diffsymbol}
\AtBeginDocument{\mathdef{\d}{\diff}}
\newderivative{\derivative}{\diffsymbol}
\newcommand\dv{\derivative}
%    \end{macrocode}
% \end{macro}
% \end{macro}
% \end{macro}
% \end{macro}
% \end{macro}
%
% \begin{macro}{\partialdifferential}
% \begin{macro}{\pd}
% \begin{macro}{\partialderivative}
% \begin{macro}{\pdv}
% Define the partial differential and derivative.
%    \begin{macrocode}
\newcommand\partialdifferential{\differential\partial}
\newcommand\slashedpartialdifferential{\differential{\slashed\partial}}
\newcommand\spd{\slashedpartialdifferential}
\newcommand\pd{\partialdifferential}
\newpartialderivative{\partialderivative}{\partial}
\newcommand\pdv{\partialderivative}
%    \end{macrocode}
% \end{macro}
% \end{macro}
% \end{macro}
% \end{macro}
%
% \begin{macro}{\gaugediffsymbol}
% \begin{macro}{\gaugediff}
% \begin{macro}{\D}
% Define the gauge covariant differential "\D".
%    \begin{macrocode}
\providecommand{\gaugediffsymbol}{D}
\newcommand{\gaugediff}{\differential\gaugediffsymbol}
\newcommand{\D}{\gaugediff}
%    \end{macrocode}
% \end{macro}
% \end{macro}
% \end{macro}
%
% \begin{macro}{\covariantdiff}
% \begin{macro}{\cd}
% Define the covariant differential "\cd".
%    \begin{macrocode}
\newcommand{\covariantdiff}{\differential\nabla}
\newcommand{\cd}{\covariantdiff}
%    \end{macrocode}
% \end{macro}
% \end{macro}
%
% \begin{macro}{\variation}
% \begin{macro}{\var}
% \begin{macro}{\functionalderivative}
% \begin{macro}{\fdv}
% Define the functional variation and derivative.
%    \begin{macrocode}
\newcommand\variation{\differential\delta}
\newcommand\var{\variation}
\newpartialderivative{\functionalderivative}{\delta}
\newcommand\fdv{\functionalderivative}
%    \end{macrocode}
% \end{macro}
% \end{macro}
% \end{macro}
% \end{macro}
%
% \begin{macro}{\cancel}
% \begin{macro}{\slashed}
% Load the \software{cancel} \cite{cancel} and \software{slashed} \cite{slashed} packages which provide the "\cancel" and "\slashed" macros.
%    \begin{macrocode}
\RequirePackage{cancel}
\RequirePackage{slashed}
\declareslashed{}{/}{.14}{0}{L}
\declareslashed{}{/}{.055}{0}{\partial}
\declareslashed{}{/}{.06}{0}{\gaugediff}
\declareslashed{}{/}{.055}{0}{\partialdifferential}
%    \end{macrocode}
% \end{macro}
% \end{macro}
%
% \subsection{Paired delimiters}
%
% \begin{macro}{\left}
% \begin{macro}{\right}
% Load the \software{mleftright} package \cite{mleftright} and adjust the spacing around "\left" and "\right".
%    \begin{macrocode}
\RequirePackage{mleftright}
\mleftright
%    \end{macrocode}
% \end{macro}
% \end{macro}
%
% \begin{macro}{\noargumentsymbol}
% \begin{macro}{\optionalargument}
% Allow for macros to have an empty argument using the \software{etoolbox} package \cite{etoolbox}.
%    \begin{macrocode}
\RequirePackage{etoolbox}
\newcommand{\noargumentsymbol}{\:\cdot\:}
\newcommand{\optionalargument}[1]{\ifblank{#1}{\noargumentsymbol}{#1}}
%    \end{macrocode}
% \end{macro}
% \end{macro}
%
% \begin{macro}{\abs}
% \begin{macro}{\norm}
% Absolute value and norm.
%    \begin{macrocode}
\DeclarePairedDelimiterX\abs[1]\lvert\rvert{\optionalargument{#1}}
\DeclarePairedDelimiterX\hep@norm[1]\lVert\rVert{\optionalargument{#1}}
\DeclarePairedDelimiterXPP\hep@pnorm[2]{}\lVert\rVert{_{#1}}{#2}
\NewDocumentCommand{\norm}{som}{%
  \IfValueTF{#2}{%
    \IfBooleanTF{#1}{\hep@pnorm*}{\hep@pnorm}{#2}%
  }{%
    \IfBooleanTF{#1}{\hep@norm*}{\hep@norm}%
  }{\optionalargument{#3}}%
}
%    \end{macrocode}
% \end{macro}
% \end{macro}

% \begin{macro}{\floor}
% \begin{macro}{\ceil}
% Floor and ceiling paired delimiters.
%    \begin{macrocode}
\DeclarePairedDelimiter\ceil{\lceil}{\rceil}
\DeclarePairedDelimiter\floor{\lfloor}{\rfloor}
%    \end{macrocode}
% \end{macro}
% \end{macro}

% \begin{macro}{\ordersymbol}
% \begin{macro}{\order}
% Order symbol and macro.
%    \begin{macrocode}
\providecommand{\ordersymbol}{\mathcal{O}}
\DeclarePairedDelimiterXPP\order[1]{\ordersymbol}(){}{#1}
%    \end{macrocode}
% \end{macro}
% \end{macro}

% \begin{macro}{\evaluated}
% \begin{macro}{\eval}
% Vertical evaluation bar
%    \begin{macrocode}
\DeclarePairedDelimiter{\hep@evaluated}{.}{\rvert}
\NewDocumentCommand{\evaluated}{som}{%
  \IfBooleanTF{#1}{%
    \hep@evaluated*{#3}%
  }{%
    \IfNoValueTF{#2}{#3\rvert}{\hep@evaluated[#2]{#3}}%
  }%
}
\newcommand\eval{\evaluated}
%    \end{macrocode}
% \end{macro}
% \end{macro}

% \begin{macro}{\row}
% \begin{macro}{\column}
% Shortcuts for rows and columns
%    \begin{macrocode}
\newcommand*{\rowseperator}{,\,}
\ExplSyntaxOn
\newcommand*{\hep@row}[1]{
 \seq_set_split:Nnn\hep@seq{,}{#1}
 \begin{matrix}\seq_use:Nn\hep@seq{\rowseperator}\end{matrix}
}
\newcommand*{\hep@column}[1]{%
 \seq_set_split:Nnn\hep@seq{,}{#1}%
 \begin{matrix}\seq_use:Nn\hep@seq{\\}\end{matrix}%
}
\ExplSyntaxOff
\DeclarePairedDelimiterX{\row}[1]{(}{)}{\hep@row{#1}}
\NewDocumentCommand{\column}{me{^}e{_}}{%
  \left(\hep@column{#1}\right)%
  \IfValueT{#2}{^{\!\!\!#2}}\IfValueT{#3}{_{\!\!\!#3}}%
}
%    \end{macrocode}
% \end{macro}
% \end{macro}

% \subsubsection{Set and Probability}
%
% \begin{macro}{\midbar}
% Define a generic midbar.
%    \begin{macrocode}
\newcommand\hep@left@delim{\mathopen{}}
\providecommand{\midbar}[1][]{%
  \nonscript\:#1\vert\allowbreak\nonscript\:\hep@left@delim%
}
%    \end{macrocode}
% \end{macro}

% Check if "nfssect-cfr" is loaeded and patch the global "\set" macro into the "cfr" namespace
%    \begin{macrocode}
\RequirePackage{xpatch}
\@ifundefined{exfs@merge@families}{}{%
  \xpatchcmd{\exfs@merge@families}{\set}{\cfr@set}{}{}%
  \xpatchcmd{\exfs@merge@families}{\set}{\cfr@set}{}{}%
  \xpatchcmd{\exfs@merge@families}{\set}{\cfr@set}{}{}%
}%
%    \end{macrocode}
% \begin{macro}{\suchthat}
% \begin{macro}{\set}
% Define a "\set" macro that allows a midbar via "\suchthat".
%    \begin{macrocode}
\providecommand\suchthat{\midbar}
\DeclarePairedDelimiterX\set[1]\{\}{%
  \renewcommand\suchthat{\midbar[\delimsize]}#1%
}
%    \end{macrocode}
% \end{macro}
% \end{macro}

% \begin{macro}{\probabilitysymbol}
% \begin{macro}{\given}
% \begin{macro}{\Pr}
% Redefine the "\Pr" macro to a macro that takes a "\given" macro and generates a midbar.
%    \begin{macrocode}
\providecommand{\probabilitysymbol}{\operatorname{Pr}}
\providecommand\given{\midbar}
\DeclarePairedDelimiterXPP\hep@Pr[1]{%
  \probabilitysymbol}(){}{%
  \renewcommand\given{\midbar[\delimsize]}#1%
}
\let\Pr\relax
\NewDocumentCommand{\Pr}{so}{%
  \IfValueTF{#2}{%
    \IfBooleanTF{#1}{\hep@Pr*}{\hep@Pr}{#2}%
  }{%
    \probabilitysymbol%
  }%
}
%    \end{macrocode}
% \end{macro}
% \end{macro}
% \end{macro}

% \subsubsection{Commutators}
%
% \begin{macro}{\newpair}
% Define the "\newpair" macro that generates pairs surrounded by brackets.
%    \begin{macrocode}
\NewDocumentCommand{\newpair}{mmme{_}e{^}}{%
  \IfNoValueTF{#4}{%
    \IfNoValueTF{#5}{%
      \DeclarePairedDelimiterX{#1}[2]{#2}{#3}%
    }{%
      \DeclarePairedDelimiterXPP{#1}[2]{}{#2}{#3}{^{#5}}%
    }%
  }{%
    \DeclarePairedDelimiterXPP{#1}[2]{}{#2}{#3}{_{#4}}%
  }{%
    \optionalargument{##1},\optionalargument{##2}%
  }%
}
%    \end{macrocode}
%\end{macro}
%
% \begin{macro}{\innerproduct}
% \begin{macro}{\poissonbracket}
% \begin{macro}{\pb}
% \begin{macro}{\commutator}
% \begin{macro}{\comm}
% \begin{macro}{\anticommutator}
% \begin{macro}{\acomm}
% Poissonbracket, commutator and anti-commutator.
%    \begin{macrocode}
\newpair\innerproduct\langle\rangle
\newpair\poissonbracket\lbrace\rbrace
\newpair\commutator\lbrack\rbrack
\newcommand\pb{\poissonbracket}
\newcommand\comm{\commutator}
\newcommand\acomm{\poissonbracket}
%    \end{macrocode}
% \end{macro}
% \end{macro}
% \end{macro}
% \end{macro}
% \end{macro}
% \end{macro}
% \end{macro}
%
% \subsubsection{Bra-ket notation}
%
% \begin{macro}{\braketspace}
% Define the space within braket notation.
%    \begin{macrocode}
% \providecommand\braketspace{\mskip1mu}
\providecommand\braketouterspace{\mskip1mu}
\providecommand\braketinnerspace{\mskip3mu}
\newcommand\hep@midvert{%
  \braketinnerspace\delimsize\vert\braketinnerspace\hep@left@delim%
}
%    \end{macrocode}
% \end{macro}
%
% \begin{macro}{\braket}
% Define the braket macro.
%    \begin{macrocode}
\DeclarePairedDelimiterX\braket[2]{\langle}{\rangle}{%
  \braketouterspace#1\hep@midvert#2\braketouterspace%
}
%    \end{macrocode}
% \end{macro}

% \begin{macro}{\bra}
% Define the bra macro.
%    \begin{macrocode}
\DeclarePairedDelimiterXPP\hep@bra[1]{%
  }{\langle}{\rvert}{\braketinnerspace%
  }{\braketouterspace#1\braketinnerspace%
}
\NewDocumentCommand{\bra}{smt\ket sgt\ketbra sgg}{%
  \IfBooleanTF{#6}{%
    \IfBooleanTF{#1}{\braket*{#2}{#8}}{\braket{#2}{#8}}%
    \IfBooleanTF{#7}{\bra*{#9}}{\bra{#9}}%
  }{
    \IfBooleanTF{#3}{%
      \IfBooleanTF{#1}{\braket*}{%
        \IfBooleanTF{#4}{\braket*}{\braket}}{#2}{#5%
      }%
    }{%
      \IfBooleanTF{#1}{\hep@bra*}{\hep@bra}{#2}%
    }%
  }%
}
%    \end{macrocode}
% \end{macro}

% \begin{macro}{\ket}
% Define the ket macro.
%    \begin{macrocode}
\DeclarePairedDelimiterXPP\ket[1]{%
  \braketinnerspace}{\lvert}{\rangle}{%
}{%
  \braketinnerspace\hep@left@delim#1\braketouterspace%
}
%    \end{macrocode}
% \end{macro}

% \begin{macro}{\ketbra}
% Define the ketbra macro.
%    \begin{macrocode}
\NewDocumentCommand{\ketbra}{smm}{%
  \IfBooleanTF{#1}{%
    \ket*{#2}\bra*{#3}%
  }{%
    \ket{#2}\bra{#3}%
  }%
}
%    \end{macrocode}
% \end{macro}

% \begin{macro}{\matrixelement}
% \begin{macro}{\mel}
% Define the matrixelement macro.
%    \begin{macrocode}
\DeclarePairedDelimiterX\matrixelement[3]{%
  \langle}{\rangle
}{%
  \braketouterspace#1\hep@midvert#2\hep@midvert#3\braketouterspace%
}
\newcommand\matrixel{\matrixelement}
\newcommand\mel{\matrixelement}
%    \end{macrocode}
% \end{macro}
% \end{macro}

% \begin{macro}{\expectationvalue}
% \begin{macro}{\ev}
% \begin{macro}{\vev}
% Define the expectationvalue and vev macros.
%    \begin{macrocode}
\DeclarePairedDelimiterX\hep@expvalue[1]{\langle}{\rangle}{%
  \braketouterspace#1\braketouterspace%
}
\NewDocumentCommand{\expectationvalue}{som}{%
  \IfNoValueTF{#2}{%
    \IfBooleanTF{#1}{\hep@expvalue*}{\hep@expvalue}{#3}%
  }{%
    \IfBooleanTF{#1}{\matrixelement*}{\matrixelement}{#2}{#3}{#2}%
  }%
}
\newcommand\ev{\expectationvalue}
\newcommand\vev[1]{\expectationvalue[0]{#1}}
%    \end{macrocode}
% \end{macro}
% \end{macro}
% \end{macro}
%
%</package>
%
% \section{Test}
%
%<*test>
%
%    \begin{macrocode}
\documentclass{article}

\usepackage{hep-math}

\begin{document}

\begin{gather}
  \bra{x}\ket{y}
  \braket*{x}{y}\\
  \dv[f]{x}^3
  \pdv[f]{x}[y]^2[z]^3
  \fdv[f]{x}^3[y]\\
  \set{x \suchthat x \in X}
\end{gather}

\end{document}

%    \end{macrocode}
%</test>
%
% \section{Readme}
%
%<*readme>
%
%    \begin{macrocode}
# The `hep-math` package

Extended math macros

## Introduction

The `hep-math` package provides some additional features beyond the
`mathtools` and `amsmath` packages.

To use the package place `\usepackage{hep-math}` in the preamble.

## Author

Jan Hajer

## License

This file may be distributed and/or modified under the conditions of the
`LaTeX` Project Public License, either version 1.3c of this license or
(at your option) any later version. The latest version of this license is
in `http://www.latex-project.org/lppl.txt` and version 1.3c or later is
part of all distributions of LaTeX version 2005/12/01 or later.
%    \end{macrocode}
%
%</readme>
%
% \Finale

\endinput

% \PrintIndex
% makeindex -s gglo.ist -o hep-math-implementation.gls hep-math-implementation.glo
% makeindex -s gglo.ist -o hep-math-implementation.ind hep-math-implementation.idx
