% \iffalse meta-comment
%
% Copyright (C) 2019-2023 by Jan Hajer
% -----------------------------------
%
% This file may be distributed and/or modified under the
% conditions of the LaTeX Project Public License, either version 1.3c
% of this license or (at your option) any later version.
% The latest version of this license is in:
%
% http://www.latex-project.org/lppl.txt
%
% and version 1.3c or later is part of all distributions of LaTeX
% version 2005/12/01 or later.
%
% \fi
%
% \iffalse

%<package>\NeedsTeXFormat{LaTeX2e}[2005/12/01]
%<package>\ProvidesPackage{hep-talk}[2024/11/01 v1.3 Talks in High Energy Physics]
%<dark>\ProvidesPackage{hep-dark}[2024/11/01 v1.3 Dark color scheme]
%<ligt>\ProvidesPackage{hep-light}[2024/11/01 v1.3 Light color scheme]
%<blue>\ProvidesPackage{hep-blue}[2024/11/01 v1.3 Blue color scheme]
%<metropolis>\ProvidesPackage{hep-metropolis}[2024/11/01 v1.3 Metropolis color scheme]
%
%<*documentation>
\RequirePackage[l2tabu, orthodox]{nag}
\documentclass{ltxdoc}

\renewcommand*\theCodelineNo{\rmfamily\tstyle\footnotesize\arabic{CodelineNo}}
\AtBeginEnvironment{macrocode}{\renewcommand*{\ttdefault}{clmt}}
\renewcommand*{\MacroFont}{\codestyle}
\AtBeginDocument{\DeleteShortVerb{\|}}
\AtBeginDocument{\MakeShortVerb{\"}}
\EnableCrossrefs
\CodelineIndex
\RecordChanges

\RequirePackage{hologo}

\RequirePackage[parskip,oldstyle,font=10pt]{hep-paper}
\bibliography{bibliography}
%</documentation>

%<*driver>
\expandafter\newif\csname ifshort\endcsname
\shortfalse
\begin{document}
\DocInput{hep-talk-implementation.dtx}
\end{document}
%</driver>
%
% \fi
%
% \CheckSum{642}
%
% \CharacterTable
%  {Upper-case    \A\B\C\D\E\F\G\H\I\J\K\L\M\N\O\P\Q\R\S\T\U\V\W\X\Y\Z
%   Lower-case    \a\b\c\d\e\f\g\h\i\j\k\l\m\n\o\p\q\r\s\t\u\v\w\x\y\z
%   Digits        \0\1\2\3\4\5\6\7\8\9
%   Exclamation   \!     Double quote  \"     Hash (number) \#
%   Dollar        \$     Percent       \%     Ampersand     \&
%   Acute accent  \'     Left paren    \(     Right paren   \)
%   Asterisk      \*     Plus          \+     Comma         \,
%   Minus         \-     Point         \.     Solidus       \/
%   Colon         \:     Semicolon     \;     Less than     \<
%   Equals        \=     Greater than  \>     Question mark \?
%   Commercial at \@     Left bracket  \[     Backslash     \\
%   Right bracket \]     Circumflex    \^     Underscore    \_
%   Grave accent  \`     Left brace    \{     Vertical bar  \|
%   Right brace   \}     Tilde         \~}
%
% \changes{v1.0}{2019/01/01}{Initial version of the style file}
% \changes{v1.1}{2022/11/01}{Bug fixes.}
% \changes{v1.2}{2023/07/01}{Bug fixes.}
% \changes{v1.3}{2024/11/01}{Bug fixes.}
%
% \ifshort
%<*documentation>
% \fi
%
\GetFileInfo{hep-talk.sty}

\title{The \software{hep-talk} package\thanks{This document corresponds to \software{hep-talk}~\fileversion.}}
\subtitle{Easy and beautiful slides}
\author{Jan Hajer \email{jan.hajer@tecnico.ulisboa.pt}}
\date{\filedate}

% \ifshort
\begin{document}
% \fi

\newgeometry{vscale=.8, vmarginratio=3:4, includeheadfoot, left=11em, marginparwidth=4.6cm, marginparsep=3mm, right=7em}

\maketitle

\begin{abstract}
The \software{hep-talk} package provides macros for the simple generation of slides for talks.
While the package provides significant fewer features than the \software{beamer} package an emphasis is placed on automatically perfect alignment of the content.
\end{abstract}
%
\section{Introduction}

The package can be loaded using "\RequirePackage{hep-talk}".

\DescribeMacro{paperwidth}
\DescribeMacro{ratio}
The size of the slides is adjusted with the "paperwidth="\meta{dimension} option.
The ratio can be adjusted with the "ratio="\meta{fraction} option.

\section{Macros and environments}

\DescribeMacro{\newframe}
New frames can be generated using the "\newframe"\oarg{title} macro.

\DescribeEnv{block}
The basic building block is the "\begin"\marg{block}\oarg{title} environment, that generates a block with attached title stretching over the whole page.

\DescribeEnv{columns}
The "\begin"\marg{columns}\oarg{title}\oarg{column} environment, generates space for \meta{column} columns if \meta{column} is integer and a single column with width \meta{columm}"\linewidth" if \meta{column} is smaller than one.

\DescribeMacro{\column}
The  "\column"\oarg{title} and "\column"\oarg{title}\marg{size} macros generates a new column.

\DescribeMacro{\row}
The "\row"\oarg{title} macro starts a new row in a "block" or "column".

% \ifshort
\printbibliography

\end{document}
%
%</documentation>
% \fi
%
% \StopEventually{
% \printbibliography
% \PrintChanges
% }
%
% \appendix
%
% \section{Implementation}
%
%<*package>
%
% \section{Options}
%
% Load the \software{kvoptions} package \cite{kvoptions} and define a "heptalk" namespace.
%    \begin{macrocode}
\RequirePackage{kvoptions}
\SetupKeyvalOptions{
  family=heptalk,
  prefix=heptalk@
}
%    \end{macrocode}
%
% \begin{macro}{paperwidth}
% Define a "paperwidth="\meta{dimension} option.
% Default to \unit[16]{cm}.
% \DoNotIndex{\DeclareStringOption}
%    \begin{macrocode}
\DeclareStringOption[16cm]{paperwidth}
%    \end{macrocode}
% \end{macro}
%
% \begin{macro}{ratio}
% Define a "ratio="\meta{fraction} option.
% Default to 3/4.
%    \begin{macrocode}
\DeclareStringOption[3/4]{ratio}
%    \end{macrocode}
% \end{macro}

% \subsection{Process options}
%
%    \begin{macrocode}
\ProcessKeyvalOptions*
%    \end{macrocode}
%
% Load the \software{hep-paper} package \cite{hep-paper} and reset the paper geometry.
%    \begin{macrocode}
% \RequirePackage[sansserif, bibliography=authoryear]{hep-paper}
\RequirePackage[oldstyle, sans]{hep-font}
\RequirePackage{hep-math-font}
\RequirePackage{hep-bibliography}
\RequirePackage{hep-title}
% \RequirePackage{hep-float}
\RequirePackage{hep-text}
% \RequirePackage{hep-reference}
% \RequirePackage{hep-acronym}
%    \end{macrocode}
% Reset the paper geometry.
%    \begin{macrocode}
\RequirePackage{geometry}
\geometry{
  reset,
  paperwidth=\heptalk@paperwidth,
  paperheight=\heptalk@paperwidth*\heptalk@ratio,
  margin=0pt,
  ignoreheadfoot,
  bottom=.02\paperwidth
}
%    \end{macrocode}

% \section{Lengths}
%
% Set internal lengths.
% Horizontal lengths.
%    \begin{macrocode}
\newlength{\hmargin}
\setlength{\hmargin}{.0125\paperwidth}
\setlength{\leftskip}{\hmargin}
\setlength{\parindent}{0pt}
%    \end{macrocode}
% Vertical lengths
%    \begin{macrocode}
\newlength{\vmargin}
\setlength{\vmargin}{%
  .01\paperheight plus .1\paperheight minus .005\paperheight%
}
\setlength{\parskip}{\vmargin}
%    \end{macrocode}
% Adjust spacing around equations.
%    \begin{macrocode}
\g@addto@macro\normalsize{%
  \setlength\abovedisplayskip{0pt}%
  \setlength\abovedisplayshortskip{0pt}%
  \setlength\belowdisplayskip{1ex}%
  \setlength\belowdisplayshortskip{1ex}%
}
%    \end{macrocode}

% \section{Colors}
%
% Colors managment is implemented using the \software{xcolor} package \cite{xcolor}.
% Make the \software{overlays} package \cite{overlays} compatible.
%    \begin{macrocode}
\RequirePackage{xcolor}
\newcommand*{\globalcolor}[1]{%
  \color{#1}\global\let\default@color\current@color%
}
\AtBeginDocument{
  \pagecolor{PageBg}
  \globalcolor{PageFg}
  \colorlet{background}{PageBg}
}
%    \end{macrocode}

% \section{Generic macros}

% \begin{macro}{\calc}
% \begin{macro}{\IfNoValueOrEmptyTF}
% Internal macros
%    \begin{macrocode}
\ExplSyntaxOn
  \cs_new_eq:NN \calc \fp_eval:n
  \long\def\IfNoValueOrEmptyTF#1#2#3{%
    \IfNoValueTF{#1}{#2}{\tl_if_empty:nTF{#1}{#2}{#3}}%
  }
\ExplSyntaxOff
%    \end{macrocode}
% \end{macro}
% \end{macro}

% \section{Overlays}

%
% Overlay managment is provided by the "overlays" package \cite{overlays}.
%    \begin{macrocode}
\RequirePackage{overlays}
%    \end{macrocode}
%
% \begin{macro}{\ifinoverl@yspec}
% Pactch the internal "\ifinoverl@yspec" macro to adjust the global "\maxoverl@y" counter to the maximal overlay specifier found within its loop.
%    \begin{macrocode}
\def\ifinoverl@yspec#1#2{%
  \global\let\inoverl@yspec\@secondoftwo
  \foreach \i in {#2}{%
    \afterassignment\getoverl@yspecb
    \overl@yspeca=0\i\relax
    \ifnum\maxoverl@y<\overl@yspeca%
      \global\maxoverl@y=\overl@yspeca\else%
    \fi%
    \pgfmathtruncatemacro\result{%
      (#1>=\overl@yspeca) && (#1<=\overl@yspecb)%
    }%
    \ifnum\result=1\relax
      \breakforeach
      \global\let\inoverl@yspec\@firstoftwo
    \fi%
  }%
  \inoverl@yspec%
}
%    \end{macrocode}
% \end{macro}

% \begin{macro}{overlays}
% Remove the mandatory argument from the "overlays" environment.
%    \begin{macrocode}
\RenewEnviron{overlays}{%
  \s@vecounters
  \s@veseries
  \maxoverl@y=1%
  \curoverl@y=0%
  \loop
    \advance\curoverl@y by 1%
    \begingroup
    \BODY
    \endgroup
  \ifnum\curoverl@y<\maxoverl@y%
    \clearpage
    \restores@vedcounters
    \restores@vedseries
    \repeat
}
%    \end{macrocode}
% \end{macro}
% redefine "only"
%    \begin{macrocode}
\let\hep@only\only
\RenewDocumentCommand{\only}{d<>m}{%
  \IfNoValueOrEmptyTF{#1}{#2}{\hep@only{#1}{#2}}%
}
\let\hep@visible\visible
\RenewDocumentCommand{\visible}{d<>m}{%
  \IfNoValueOrEmptyTF{#1}{#2}{\hep@visible{#1}{#2}}%
}
%    \end{macrocode}


% \section{Blocks}
%
% \begin{macro}{\hep@block@width}
% The internal block width
%    \begin{macrocode}
\newcommand*{\hep@block@width}[1]{%
  \dimexpr#1\linewidth-2\fboxsep-2\leftskip\relax%
}
%    \end{macrocode}
% \end{macro}

% \begin{macro}{\hep@block}
% The internal "\hep@block" macro.
% \begin{enumerate}[nosep, label=\#\arabic*]
%  \item mandatory star boolean switching between "visible" and "only"
%  \item overlay number
%  \item optional background color
%  \item optional forground color
%  \item optional width of the block
%  \item mandatory text
% \end{enumerate}
%    \begin{macrocode}
\NewDocumentCommand{\hep@block}{md<>O{BlockBg}O{BlockFg}O{1}m}{%
  \IfNoValueOrEmptyTF{#2}{}{\IfBooleanTF{#1}{\visible<#2>}{\only<#2>}}%
  \bgroup\textcolor{#4}{\colorbox{#3}{%
  \begin{minipage}[t]{\hep@block@width{#5}}%
    #6\strut%
  \end{minipage}%
  }}\egroup%
}
%    \end{macrocode}
% \end{macro}

% \begin{macro}{hep@block@env}
% The internal "hep@block@env" environment.
% \begin{enumerate}[nosep, label=\#\arabic*]
%  \item mandatory star boolean switching between "visible" and "only"
%  \item overlay number
%  \item optional background color
%  \item optional forground color
%  \item optional width of the block
% \end{enumerate}
%    \begin{macrocode}
\newsavebox{\hep@saved@box}
\NewDocumentEnvironment{hep@block@env}{%
  md<>O{BlockBg}O{BlockFg}O{1}%
}{%
  \IfNoValueOrEmptyTF{#2}{}{%
    \IfBooleanTF{#1}{\visible<#2>}{\only<#2>}%
  }%
  \bgroup\color{#4}%
  \begin{lrbox}{\hep@saved@box}%
  \begin{minipage}{\hep@block@width{#5}}%
}{%
  \end{minipage}\end{lrbox}%
  \colorbox{#3}{\usebox{\hep@saved@box}}%
  \egroup%
}
%    \end{macrocode}
% \end{macro}

% \subsection{Public Blocks}
%
% \begin{macro}{block}
% The basic "block" environment.
% \begin{enumerate}[nosep, label=\#\arabic*]
%  \item star switching the "overlay" from "only" to "visible"
%  \item overlay number
%  \item optional name of the block
%  \item optional width of the block
% \end{enumerate}
%    \begin{macrocode}
\providecommand*{\row}{}
\NewDocumentEnvironment{block}{sd<>oO{1}}{%
  \NewDocumentCommand{\hep@begin@block@env}{md<>oO{1}}{%
    \IfNoValueOrEmptyTF{##3}{}{%
      \hep@block##1<##2>[CaptionBg][CaptionFg][##4]{##3}%
      \setlength{\parskip}{0pt}\par\vspace{-1pt}%
    }%
    \begin{hep@block@env}##1<##2>[BlockBg][BlockFg][##4]%
  }%
  \RenewDocumentCommand{\row}{sd<>oO{1}}{%
    \hep@end@block@env%
    \hep@begin@block@env##1<##2>[##3][##4]%
  }%
  \hep@begin@block@env#1<#2>[#3][#4]%
}{\hep@end@block@env}
%    \end{macrocode}
% \end{macro}

% \begin{macro}{\hep@end@block@env}
% The internal "\hep@end@block@env" macro.
%    \begin{macrocode}
\newcommand*{\hep@end@block@env}{%
  \end{hep@block@env}\setlength{\parskip}{\vmargin}\par%\par%
}
%    \end{macrocode}
% \end{macro}

% \begin{macro}{columns}
% The "columns" environment.
% \begin{enumerate}[nosep, label=\#\arabic*]
%  \item star switching the "overaly" from "only" to "visible"
%  \item overlay number
%  \item optional name of the first column
%  \item optional width of the first column
% \end{enumerate}
%    \begin{macrocode}
\providecommand*{\column}{}
\NewDocumentEnvironment{columns}{sd<>oO{2}}{%
  \begin{minipage}{\linewidth-2\leftskip}%
  \def\needs@space{}%
%    \end{macrocode}
%
% \begin{macro}{\hep@mini@page}
% The private "\hep@mini@page" macro.
% \begin{enumerate}[nosep, label=\#\#\arabic*]
%  \item mandatory star argument
%  \item overlay number
%  \item optional width of the minipage
%  \item mandatory name of the minipage
% \end{enumerate}
%    \begin{macrocode}
  \NewDocumentCommand{\hep@mini@page}{md<>O{#4}m}{%
    \begin{minipage}[t]{%
      \dimexpr##3\linewidth+##3\hmargin-\hmargin-.01pt\relax%
    }%
    \vspace{0pt}%
    \IfBooleanTF{##1}{\begin{block}*<##2>[##4][1]}{%
      \IfNoValueOrEmptyTF{##2}{}{\gdef\needs@space{##2}}%
      \begin{block}<##2>[##4][1]%
    }%
  }%
%    \end{macrocode}
% \end{macro}
%
% check if the column width is larger than 1.
%    \begin{macrocode}
  \ifdim#4pt>1pt%
    \def\hep@inverse{\calc{1/#4}}%
%    \end{macrocode}
%
% \begin{macro}{\column}
% The public "\column" macro.
% \begin{enumerate}[nosep, label=\#\#\arabic*]
%  \item star switching the "overlay" from "only" to "visible"
%  \item overlay number
%  \item optional name
% \end{enumerate}
%    \begin{macrocode}
    \RenewDocumentCommand{\column}{sd<>oo}{%
      \end{block}\end{minipage}%
      \ifdefempty{\needs@space}{\hspace{\hmargin}}{%
        \only<\needs@space>{\hspace{\hmargin}}%
      }%
      \gdef\needs@space{}%
      \hep@mini@page##1<##2>[%
        \IfNoValueOrEmptyTF{##4}{\hep@inverse}{\calc{1/##4}}%
      ]{##3}%
    }%
%    \end{macrocode}
% \end{macro}
%
% start a new minipage.
% The case that column width is smaller than 1.
%    \begin{macrocode}
    \hep@mini@page#1<#2>[\hep@inverse]{#3}%
  \else%
%    \end{macrocode}
%
% \begin{macro}{\column}
% The public "\column" macro.
% \begin{enumerate}[nosep, label=\#\#\arabic*]
%  \item star switching the "overlay" from "only" to "visible"
%  \item overlay number
%  \item optional name
%  \item mandatory width
% \end{enumerate}
%    \begin{macrocode}
    \RenewDocumentCommand{\column}{sd<>oo}{%
      \end{block}\end{minipage}%
      \ifdefempty{\needs@space}{\hspace{\hmargin}}{%
        \only<\needs@space>{\hspace{\hmargin}}%
      }%
      \gdef\needs@space{}%
      \hep@mini@page##1<##2>[%
        \IfNoValueOrEmptyTF{##4}{#4}{##4}%
      ]{##3}%
    }%
%    \end{macrocode}
% \end{macro}
%
% start a new minipage; end of check for argument size and end of the columns environment.
%    \begin{macrocode}
    \hep@mini@page#1<#2>[#4]{#3}%
  \fi%
}{\end{block}\end{minipage}\end{minipage}\par}%
%    \end{macrocode}
% \end{macro}
%
%    \begin{macrocode}
\RequirePackage{hep-math}
%    \end{macrocode}

% \section{Frames}
%
% \begin{macro}{\newframe}
% The "\newframe" macro
%    \begin{macrocode}
\NewDocumentCommand{\newframe}{o}{%
  \clearpage%
  \IfNoValueF{#1}{%
    \leftskip=0pt%
    \hep@block\BooleanFalse[TitleBg][TitleFg][1]{%
      \hspace{\dimexpr\hmargin\relax}\LARGE\strut\Large#1\hspace{\hmargin}%
    }%
  }\par\leftskip=\hmargin%
}
%    \end{macrocode}
% \end{macro}

% \begin{macro}{slide}
% Define a "slide" macro that starts a new frame with overlays.
%    \begin{macrocode}
\newenvironment{slide}[1][]{%
  \overlays%
  \newframe[#1]%
}{%
  \endoverlays%
}
%    \end{macrocode}
% \end{macro}

% \begin{macro}{robustslide}
% Define a "slide" macro that starts a new frame with overlays.
% TODO redefine the "robustoverlays" macro.
%    \begin{macrocode}
\newenvironment{robustslide}[1][]{%
  \robustoverlays%
  \newframe[#1]%
}{%
  \endrobustoverlays%
}
%    \end{macrocode}
% \end{macro}


% \begin{macro}{\logo}
% The "\logo" macro
%    \begin{macrocode}
\renewcommand*\Affilfont{\normalsize}
\newcommand*{\titlemargin}{.05\paperwidth}
\newcommand*{\prelogo}[1]{\def\hep@pre@logo{#1}}
\newcommand*{\logo}[1]{\def\hep@logo{#1}}
\newcommand*{\postlogo}[1]{\def\hep@post@logo{#1}}
\prelogo{\vfill\hfill}
\postlogo{}
%    \end{macrocode}
% \end{macro}

% \begin{macro}{\maketitle}
% The "\maketitle" macro
%    \begin{macrocode}
\renewcommand*{\maketitle}{%
  \newframe%
  \strut\par\vspace{10ex}\bgroup%
    \leftskip=\titlemargin
    \rightskip=\titlemargin
    {\Huge\hep@title@font\thetitle\par}%
    \vspace{.5ex}%
    \@ifundefined{sub@title}{}{\Large\hep@subtitle@font\sub@title}\par%
    \vspace{3ex}%
    {\huge\hep@author@font\@author\vspace{1ex}\par}
    \vspace{3ex}%
    \normalsize\hep@date@font\@date\par%
    \vspace{3ex}%
    \@ifundefined{hep@logo}{}{%
      \hep@pre@logo\hep@logo\hep@post@logo\par%
    }%
  \egroup%
}
%    \end{macrocode}
% \end{macro}
%
% \begin{macro}{\section}
% The "\section" macro is adjusted using the "titlesec" package \cite{titlesec}
%    \begin{macrocode}
\RequirePackage{titlesec}
\newcommand*{\sectionbreak}{\clearpage}
\titleformat{\section}{\huge}{}{0pt}{%
  \strut\vfill\centering\color{PageFg}%
}[\vfill\strut\clearpage]
%    \end{macrocode}
% \end{macro}

% \section{Content}
%
% The pagenumber is implemented using the "picture" \cite{picture} and "atbegshi" \cite{atbegshi} packages.
%    \begin{macrocode}
% \RequirePackage{picture}
% \RequirePackage{atbegshi}
\RequirePackage{tikz}
\usetikzlibrary{fadings}
\usetikzlibrary{external}
\tikzfading[
  name=fade out,
  inner color=transparent!0,
  outer color=transparent!80,
]
\tikzset{
  pagenumber/.style={
    font=\footnotesize,
    anchor=south east,
    outer sep=1ex,
    circle,
    inner sep=1pt,
    text=Structure,
    fill=PageBg,
    path fading=fade out,
  },
}
\AtBeginShipout{\AtBeginShipoutUpperLeftForeground{%
  \ifnum\thepage>1%
%     \put(\paperwidth-\widthof{\footnotesize\thepage}-\hmargin,%
%         -\paperheight+\hmargin){%
%       \footnotesize\color{Structure}\thepage%
%     }%
    \tikzset{external/export next=false}%
    \begin{tikzpicture}[remember picture,overlay]
      \node[pagenumber] at (current page.south east){\thepage};%
    \end{tikzpicture}
  \fi%
}}
%    \end{macrocode}
%
% \begin{macro}{itemize}
% \begin{macro}{description}
% Lists
%    \begin{macrocode}
\setlist[itemize]{nosep, left=0pt, label=\color{Structure}\textbullet}
\setlist[description]{nosep, font=\color{Structure}\normalfont}
\setlist[enumerate]{nosep, font=\color{Structure}\arabic*}
%    \end{macrocode}
% \end{macro}
% \end{macro}

% \begin{macro}{\emph}
% Empheses
%    \begin{macrocode}
\renewcommand*{\emph}[1]{\textcolor{Structure}{#1}}
%    \end{macrocode}
% \end{macro}
% Equations
%    \begin{macrocode}
\mathtoolsset{showonlyrefs}
%    \end{macrocode}

% \section{Bibliography}
%
% Prepare the bibliography
%    \begin{macrocode}
% \ExecuteBibliographyOptions{
%   maxcitenames=1,
%   maxbibnames=5,
%   dashed=false,
%   uniquename=false,
%   uniquelist=false
% }
% \DeclareNameAlias{sortname}{given-family}
%    \end{macrocode}
% \begin{macro}{cite}
% Add the collaboration field to the citation macro.
%    \begin{macrocode}
% \renewbibmacro*{cite}{%
%   \DeclareFieldFormat{eprint}{##1}%
%   \iffieldundef{shorthand}{%
%     \iffieldundef{eprint}{%
%       \ifthenelse{%
%         \ifnameundef{labelname}\OR\iffieldundef{labelyear}%
%        }{%
%         \usebibmacro{cite:label}%
%         \setunit{\printdelim{nonameyeardelim}}%
%       }{%
%         \iffieldundef{collaboration}{%
%           \printnames{labelname}%
%          }{%
%           \printfield{collaboration}%
%          }\setunit{\printdelim{nameyeardelim}}%
%       }\usebibmacro{cite:labeldate+extradate}%
%     }{\printfield{eprint}}%
%   }{\usebibmacro{cite:shorthand}}%
% }
% \renewbibmacro*{cite}{%
%   \iffieldundef{shorthand}{%
%     \ifthenelse{\ifnameundef{labelname}\OR\iffieldundef{labelyear}}{%
%       \usebibmacro{cite:label}%
%       \setunit{\printdelim{nonameyeardelim}}}{%
%         \iffieldundef{collaboration}{%
%           \printnames{labelname}%
%         }{%
%           \printfield{collaboration}%
%         }%
%         \setunit{\printdelim{nameyeardelim}}%
%       }%
%      \usebibmacro{cite:labeldate+extradate}%
%   }{\usebibmacro{cite:shorthand}}%
% }
%    \end{macrocode}
% \end{macro}
% clickable parencite
%    \begin{macrocode}
\DeclareCiteCommand{\parencite}[\mkbibbrackets]{\usebibmacro{prenote}}{%
  \usebibmacro{citeindex}%
  \printtext[bibhyperref]{\usebibmacro{cite}}%
}{\multicitedelim}{\usebibmacro{postnote}}
%    \end{macrocode}
% small rightaligned Structure colored citations
%    \begin{macrocode}
\newcommand*{\citestyle}[1]{%
  \hfill{\scriptsize\normalfont\color{Structure}#1}%
}
\renewcommand*{\cite}[1]{\citestyle{\autocite{#1}}}
%    \end{macrocode}
% do not color the titles
%    \begin{macrocode}
\DeclareFieldFormat{title}{\enquote{#1}\isdot}
\DeclareFieldFormat{booktitle}{\textit{#1}\isdot}
\DeclareFieldFormat{journaltitle}{\textit{#1}\isdot}
%    \end{macrocode}
% put references in blocks
%    \begin{macrocode}
% \renewbibmacro{begentry}{%
%   \vspace{-.3ex}%
%   \block\relax%
%   \setlength{\leftskip}{1.25em}%
%   \setlength{\parindent}{-1.25em}%
% }
%    \end{macrocode}
% close the blocks
%    \begin{macrocode}
% \renewbibmacro{finentry}{\endblock}
%    \end{macrocode}
%
%
%
%
% use eprint info for references
%    \begin{macrocode}
\newcommand*{\hep@talk@cite}{%
  \iffieldundef{shorthand}{%
    \iffieldundef{eprint}{%
      \ifthenelse{%
        \ifnameundef{labelname}\OR\iffieldundef{labelyear}%
       }{%
        \usebibmacro{cite:label}%
        \setunit{\printdelim{nonameyeardelim}}%
      }{%
        \iffieldundef{collaboration}{%
          \printnames{labelname}%
         }{%
          \printfield{collaboration}%
         }\setunit{\printdelim{nameyeardelim}}%
      }\usebibmacro{cite:labeldate+extradate}%
    }{\texttt{\printfield{eprint}}}%
  }{\usebibmacro{cite:shorthand}}%
}
%    \end{macrocode}
% use eprint info for reference labeles
%    \begin{macrocode}
\newcommand*{\hep@talk@label}{%
  \DeclareFieldFormat{eprint:arxiv}{%
    \ifhyperref{%
      \href{https://arxiv.org/\abx@arxivpath/##1}{\nolinkurl{##1}}%
    }{%
      \nolinkurl{##1}\iffieldundef{eprintclass}{}{%
        \addspace\texttt{\mkbibbrackets{\thefield{eprintclass}}}%
      }%
    }%
  }%
  \DeclareFieldFormat{eprint:github}{%
    \hep@bib@online{https://github.com/\thefield{eprintclass}/##1}{##1}%
  }%
  \iffieldundef{shorthand}{%
    \iffieldundef{eprint}{%
      \ifthenelse{%
        \ifnameundef{labelname}\OR\iffieldundef{labelyear}%
       }{%
        \setunit{\printdelim{nonameyeardelim}}%
      }{%
        \iffieldundef{collaboration}{%
          \printnames{labelname}%
         }{%
          \printfield{collaboration}%
         }\setunit{\printdelim{nameyeardelim}}%
      }
      \usebibmacro{cite:labeldate+extradate}%
    }{\usebibmacro{eprint}}%
  }{\usebibmacro{cite:shorthand}}%
}
%    \end{macrocode}
% remove eprint from entry
%    \begin{macrocode}
\newcommand*{\hep@talk@entry}{%
  \DeclareFieldFormat{doi}{}%
  \renewbibmacro{in:}{}%
  \renewbibmacro{journal+issuetitle}{}%
  \renewbibmacro{note+pages}{}%
  \renewbibmacro{organization+location+date}{}%
  \iffieldundef{eprint}{%
    \ifthenelse{%
      \ifnameundef{labelname}\OR\iffieldundef{labelyear}%
    }{}{%
      \iffieldundef{collaboration}{%
        \DeclareNameFormat{sortname}{}%
        \DeclareNameFormat{year}{}%
        \DeclareFieldFormat{cite:labeldate+extradate}{}%
        \DeclareFieldFormat{labelyear}{}%
      }{%
        \DeclareFieldFormat{collaboration}{}%
      }%
    }%
    \DeclareFieldFormat{cite:labeldate+extradate}{}%
  }{%
    \DeclareFieldFormat{eprint:arxiv}{}%
    \DeclareFieldFormat{eprint:github}{}%
  }%
}
%    \end{macrocode}
% redefine end of line in the header
%    \begin{macrocode}
\newcommand*{\hep@talk@entry@tail}{%
  \bibsentence%
  \usebibmacro{in:}%
  \usebibmacro{journal+issuetitle}%
  \newunit%
  \usebibmacro{note+pages}%
  \printunit{\addcomma\space}%
  \iftoggle{bbx:doi}{\printfield{doi}}{}%
}
%    \end{macrocode}
% redefine cite
%    \begin{macrocode}
\renewbibmacro*{cite}{%
  \DeclareFieldFormat{eprint}{##1}%
  \hep@talk@cite%
}
%    \end{macrocode}
% put bib item in a block
%    \begin{macrocode}
\renewbibmacro{begentry}{%
  \block[[\hep@talk@label]\hfill\hep@talk@entry@tail]%
  \hep@talk@entry%
}
%    \end{macrocode}
% close block
%    \begin{macrocode}
\renewbibmacro{finentry}{%
  \endblock\filbreak%
}
%    \end{macrocode}
% start a new frame for bibliography
%    \begin{macrocode}
\defbibheading{bibliography}[\refname]{%
  \newframe[#1]%
}
%    \end{macrocode}
% use small font for bibliography
%    \begin{macrocode}
\defbibenvironment{bibliography}{%
  \small%
%   start environment
}{
%   end environment
}{
%   item
}
%    \end{macrocode}
%
%</package>

% \section{Themes}
%
% \subsection{Dark colors}
%
%<*dark>
%
%    \begin{macrocode}
\definecolor{PageBg}{gray}{.3}
\definecolor{PageFg}{gray}{.9}
\definecolor{TitleBg}{gray}{.5}
\definecolor{TitleFg}{gray}{1}
\definecolor{CaptionBg}{gray}{.8}
\definecolor{CaptionFg}{gray}{.2}
\definecolor{BlockBg}{gray}{1}
\definecolor{BlockFg}{gray}{0}
\definecolor{seven}{HTML}{2980b9}
\colorlet{Structure}{seven!70!white}
\colorlet{Highlight}{red}
%    \end{macrocode}
%</dark>
%
% \subsection{Light colors}
%
%<*light>
%    \begin{macrocode}
\definecolor{PageBg}{gray}{1}
\definecolor{PageFg}{gray}{0}
\definecolor{TitleBg}{gray}{.8}
\definecolor{TitleFg}{gray}{.1}
\definecolor{CaptionBg}{gray}{.9}
\definecolor{CaptionFg}{gray}{.2}
\definecolor{BlockBg}{gray}{1}
\definecolor{BlockFg}{gray}{0}
\definecolor{seven}{HTML}{ff0000}
\colorlet{Structure}{seven!70!white}
\colorlet{Highlight}{red}
%    \end{macrocode}
%</light>
%
% \subsection{Blue}
%
%<*blue>
%    \begin{macrocode}
\definecolor{PageBg}{gray}{1}
\definecolor{PageFg}{gray}{0}
\definecolor{TitleBg}{gray}{1}
\definecolor{TitleFg}{HTML}{3333b2}
\definecolor{CaptionBg}{HTML}{262686}
\definecolor{CaptionFg}{gray}{1}
\definecolor{BlockBg}{HTML}{E9E9F3}
\definecolor{BlockFg}{gray}{0}
\definecolor{Structure}{HTML}{3333b2}
\colorlet{Highlight}{red}
%    \end{macrocode}
%</blue>
%
% \subsection{Metropolis}
%
%<*metropolis>
%    \begin{macrocode}
\definecolor{mDarkTeal}{HTML}{23373b}
\definecolor{mLightBrown}{HTML}{EB811B}

\colorlet{PageBg}{black!2}
\colorlet{PageFg}{mDarkTeal}
\colorlet{TitleBg}{mDarkTeal}
\colorlet{TitleFg}{black!2}
\colorlet{CaptionFg}{mDarkTeal}
\colorlet{CaptionBg}{PageBg!80!CaptionFg}
\colorlet{BlockBg}{CaptionBg!50!PageBg}
\colorlet{BlockFg}{mDarkTeal}
\colorlet{Structure}{mLightBrown}
\colorlet{Highlight}{mLightBrown}
%    \end{macrocode}
%</metropolis>

% \Finale
%
\endinput
