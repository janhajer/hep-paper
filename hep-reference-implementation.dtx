% \iffalse meta-comment
%
% Copyright (C) 2019-2022 by Jan Hajer
% -----------------------------------
%
% This file may be distributed and/or modified under the
% conditions of the LaTeX Project Public License, either version 1.3c
% of this license or (at your option) any later version.
% The latest version of this license is in:
%
% http://www.latex-project.org/lppl.txt
%
% and version 1.3c or later is part of all distributions of LaTeX
% version 2005/12/01 or later.
%
% \fi
%
% \iffalse
%
%<package>\NeedsTeXFormat{LaTeX2e}[2005/12/01]
%<package>\ProvidesPackage{hep-reference}[2022/11/01 v1.1 Publications in High Energy Physics]
%<documentation>\ProvidesFile{hep-reference-documentation.tex}[2022/11/01 v1.1 HEP-Paper documentation]
%
%<*documentation>
%
\RequirePackage[l2tabu, orthodox]{nag}
\documentclass{ltxdoc}
\AtBeginDocument{\DeleteShortVerb{\|}}
\AtBeginDocument{\MakeShortVerb{\"}}

\EnableCrossrefs
\CodelineIndex
\RecordChanges

\usepackage[parskip]{hep-paper}

\bibliography{bibliography}

\shortacronym{PDF}{portable document format}

\usepackage{hologo}

\MacroIndent=1.5em
\AtBeginEnvironment{macrocode}{\renewcommand{\ttdefault}{clmt}}
%</documentation>

%<*driver>
\expandafter\newif\csname ifshort\endcsname
\shortfalse
\begin{document}
\DocInput{hep-reference-implementation.dtx}
\end{document}
%</driver>
%
% \fi
%
% \CheckSum{179}
%
% \CharacterTable
%  {Upper-case    \A\B\C\D\E\F\G\H\I\J\K\L\M\N\O\P\Q\R\S\T\U\V\W\X\Y\Z
%   Lower-case    \a\b\c\d\e\f\g\h\i\j\k\l\m\n\o\p\q\r\s\t\u\v\w\x\y\z
%   Digits        \0\1\2\3\4\5\6\7\8\9
%   Exclamation   \!     Double quote  \"     Hash (number) \#
%   Dollar        \$     Percent       \%     Ampersand     \&
%   Acute accent  \'     Left paren    \(     Right paren   \)
%   Asterisk      \*     Plus          \+     Comma         \,
%   Minus         \-     Point         \.     Solidus       \/
%   Colon         \:     Semicolon     \;     Less than     \<
%   Equals        \=     Greater than  \>     Question mark \?
%   Commercial at \@     Left bracket  \[     Backslash     \\
%   Right bracket \]     Circumflex    \^     Underscore    \_
%   Grave accent  \`     Left brace    \{     Vertical bar  \|
%   Right brace   \}     Tilde         \~}
%
% \changes{v1.0}{2011/08/01}{Initial version of the style file.}
% \changes{v1.1}{2022/11/01}{Bug fixes.}
%
% \ifshort
%<*documentation>
% \fi
%
\GetFileInfo{hep-reference.sty}

\title{The \software{hep-reference} package\thanks{This document corresponds to \software{hep-reference}~\fileversion.}}
\subtitle{Adjustments of standard references, footnote, and citations}
\author{Jan Hajer \email{jan.hajer@tecnico.ulisboa.pt}}
\date{\filedate}

% \ifshort
\begin{document}
% \fi

\newgeometry{vscale=.8, vmarginratio=3:4, includeheadfoot, left=11em, marginparwidth=4.6cm, marginparsep=3mm, right=7em}

\maketitle

\begin{abstract}
The \software{hep-reference} package applies some light convenience modifications to the reference, citation and footnote macros after loading standard classes improving on the default \hologo{LaTeX} behaviour.
\end{abstract}

\section{Introduction}

The \software{hyperref} package is loaded and the links are hidden.

\DescribeMacro{\cref}
References are extended with the \software{cleveref} package \cite{cleveref}, which allows to \eg just type "\cref"\marg{key}  in order to write \enquote{figure 1}.
Furthermore, the \software{cleveref} package allows to reference multiple objects within one "\cref"\marg{key1,key2}.

\DescribeMacro{\cite}
Citations are adjusted to not start on a new line in order to avoid the repeated use of "~\cite"\marg{key}.

\DescribeMacro{\ref}
\DescribeMacro{\eqref}
\DescribeMacro{\subref}
References are also adjusted to not start on a new line.

\DescribeMacro{\footnote}
Footnotes are adjusted to swallow white space before the footnote mark and at the beginning of the footnote text.
Additionally the reference back to the text where they are called from.

% \ifshort
\printbibliography

\end{document}
%
%</documentation>
% \fi
%
% \StopEventually{
% \printbibliography
% \PrintChanges
% }
%
% \appendix
%
%<*package>
%
% \section{Implementation}
%
% Load the \software{hyperref} package \cite{hyperref} enable Unicode encoding and hide links.
%    \begin{macrocode}
\RequirePackage{hyperref}
\hypersetup{
  pdfencoding=auto, psdextra,
  linktoc=all, hidelinks, breaklinks=true,
  pdfcreator={}, pdfproducer={}
}
\AtBeginDocument{\@ifpackageloaded{doc}{\hypersetup{hidelinks}}{}}
%    \end{macrocode}
% Set the \PDF meta data according to the paper information and ensure that unnecessary information is suppressed.
%
%    \begin{macrocode}
\pdfstringdefDisableCommands{\def\varepsilon{\textepsilon}}
\pdfstringdefDisableCommands{\def\to{\textrightarrow}}
\AtBeginDocument{
  \pdfstringdefDisableCommands{\let\ensuremath\@gobble}
  \pdfstringdefDisableCommands{\let\mathsurround\@gobble}
  \pdfstringdefDisableCommands{\let\unskip\@gobble}
  \pdfstringdefDisableCommands{\let\thanks\@gobble}
  \pdfstringdefDisableCommands{\let\footnote\@gobble}
  \pdfstringdefDisableCommands{\let\\\@gobble}
}
\AtBeginDocument{\hypersetup{pdftitle={\@title}}}
\AtBeginDocument{\hypersetup{pdfauthor={\@author}}}
\AtBeginDocument{\@ifundefined{hep@key@words}{}{%
  \hypersetup{pdfkeywords={\hep@key@words}}%
}}
%    \end{macrocode}

% \subsection{Footnotes}
%
% Place a hyperlink from the footnote back to its referencing label using the \software{footnotebackref} package \cite{footnotebackref}.
%    \begin{macrocode}
\def\BackrefFootnoteTag{}
\RequirePackage{footnotebackref}
%    \end{macrocode}
%
% \begin{macro}{\footnote}
% Ensure that no spaces appear before the footmark or at the beginning of the footnote.
%    \begin{macrocode}
\let\@foot@note\footnote
\renewcommand{\footnote}[1]{\unskip\@foot@note{\ignorespaces#1}}
%    \end{macrocode}
% \end{macro}

% \subsection{References}
%
% \begin{macro}{\ref}
% Adjust "\ref"\marg{key} in order to prevent preceding line breaks.
%    \begin{macrocode}
\AtBeginDocument{
  \let\hep@ref\ref
  \renewcommand\ref{\no@break@before\hep@ref}
}
%    \end{macrocode}
% \end{macro}
%
% \begin{macro}{\eqref}
% Adjust "\eqref"\marg{key} from the \software{amsmath} package in order to prevent preceding line breaks.
%    \begin{macrocode}
\AtBeginDocument{\@ifpackageloaded{amsmath}{%
  \renewcommand\eqref{\no@break@before\labelcref}%
}{}}
%    \end{macrocode}
% \end{macro}
%
% \begin{macro}{\usecommand}
% Define the "\usecommand" macro providing an anonymous function.
%    \begin{macrocode}
\providecommand\use@command{}
\newcommand\usecommand[2][1]{%
  \renewcommand\use@command[#1]{#2}\use@command%
}
%    \end{macrocode}
% \end{macro}
%
% \begin{macro}{\textlabel}
% Define the "\textlabel" macro allowing to link to a piece of text.
%    \begin{macrocode}
\NewDocumentCommand\textlabel{O{}mO{}m}{%
  \protected@write\@auxout{}{%
    \string\newlabel{#2}{%
      {\hspace{-4pt}\protecting{#1{#4}}}{\thepage}{#4}{#2}{}%
    }%
  }%
  \hypertarget{#2}{}%
  {#3{#4}}%
}
%    \end{macrocode}
% \end{macro}
%
% \begin{macro}{\cref}
% Improve reference using the \software{cleveref} package \cite{cleveref}.
%    \begin{macrocode}
\PassOptionsToPackage{noabbrev, nameinlink}{cleveref}
\RequirePackage{cleveref}
\newcommand{\creflastconjunction}{, and\nobreakspace}
\crefname{enumi}{point}{points}
\crefname{inlinelisti}{point}{points}
%    \end{macrocode}
% \end{macro}
%
%\begin{macro}{\no@break@before}
% Define a macro able to prevent line breaks.
%    \begin{macrocode}
\newcommand\no@break@before{%
  \relax\ifvmode\else%
    \ifhmode%
      \ifdim\lastskip > 0pt%
        \relax\unskip\nobreakspace%
      \fi%
    \fi%
  \fi%
}
%    \end{macrocode}
% \end{macro}
%
% \begin{macro}{\subref}
% Adjust "\subref"\marg{key} from the \software{subcaption} package in order to prevent preceding line breaks.
%    \begin{macrocode}
\AtBeginDocument{\@ifpackageloaded{subcaption}{%
  \let\hep@subref\subref
  \renewcommand\subref{\no@break@before\hep@subref}
  \renewcommand*\subcaption@ref[2]{\begingroup%
    \caption@setoptions{sub}%
    \subcaption@reffmt\p@subref{\hep@ref#1{sub@#2}}%
  \endgroup}
%    \end{macrocode}
% \end{macro}
%
% \begin{macro}{\subcref}
% Provide the "\subcref" macro.
%    \begin{macrocode}
  \newcommand{\subcref}[1]{\cref{sub@#1}}
}{}}
%    \end{macrocode}
% \end{macro}
%
% \begin{macro}{\eqcrefname}
% Define the "\eqcrefname" macro for named equation types.
%    \begin{macrocode}
\NewDocumentCommand{\eqcrefname}{mmo}{
  \crefname{#1}{#2}{\IfValueTF{#3}{#3}{#2s}}
  \creflabelformat{#1}{(##2##1##3)}
}
%    \end{macrocode}
% \end{macro}
%
% \begin{macro}{\labelcrefrange}
% Define the missing "\labelcrefrange"\marg{key1}\marg{key2} macro.
%    \begin{macrocode}
\DeclareRobustCommand{\labelcrefrange}[2]{%
  \@crefrangenostar{labelcref}{#1}{#2}%
}
%    \end{macrocode}
% \end{macro}
%
% Adjust the title page
%    \begin{macrocode}
\AtBeginDocument{\@ifpackageloaded{hep-title}{%
  \labelcrefmultiformat{affiliation}{#2#1#3}{%
    \textsuperscript,#2#1#3}{\textsuperscript,#2#1#3%
  }{%
    \textsuperscript,#2#1#3%
  }
  \labelcrefrangeformat{affiliation}{#3#1#4\textsuperscript{--}#5#2#6}
}{}}
%    \end{macrocode}

% \subsection{Citations}
%
% \begin{macro}{\cite}
% Adjust "\cite"\marg{key} in order to prevent preceding line breaks.
%    \begin{macrocode}
\let\hep@cite\cite
\renewcommand\cite{\no@break@before\hep@cite}
%    \end{macrocode}
% \end{macro}
%
%</package>
%
% \section{Tests}
%
%<*test>
%
%    \begin{macrocode}
\documentclass{article}

\usepackage{amsmath}
\usepackage{hep-reference}
\usepackage{parskip}
\providecommand{\cref}{section \ref}

\begin{document}

\section{Section} \label{sec:Section}

Lorem ipsum dolor sit amet, consectetuer adipiscing elit. Etiam lobortis section \ref{sec:Section}

Lorem ipsum dolor sit amet, consectetuer adipiscing elit. Etiam lobortis \cref{sec:Section}

Lorem ipsum dolor sit amet, consectetuer adipiscing elit. Etiam loborti equation \eqref{eq:Equation}

Lorem ipsum dolor sit amet, consectetuer adipiscing elit. Etiam loborti reference \cite{cite}

Lorem ipsum dolor sit amet, consectetuer adipiscing elit.
\footnote{
 Lorem ipsum dolor sit amet, consectetuer adipiscing elit.
}

\begin{equation} \label{eq:Equation}
E=mc^2
\end{equation}


\begin{thebibliography}{1}
\bibitem{cite} reference
\end{thebibliography}

\end{document}
%    \end{macrocode}
%
%</test>
%
% \section{Readme}
%
%<*readme>
%
%    \begin{macrocode}
# The `hep-reference` package

A `LaTeX` package for publications in High Energy Physics.

## Introduction

The `hep-reference` package aims to provide a single style file containing most configurations and macros necessary to write appealing publications in High Energy Physics.
Instead of reinventing the wheel by introducing newly created macros `hep-reference` preferably loads third party packages as long as they are lightweight enough.

For usual publications it is enough to load additionally to the `article` class without optional arguments only the `hep-reference` package.

    \documentclass{article}
    \usepackage{hep-reference}

## Author

Jan Hajer

## License

This file may be distributed and/or modified under the conditions of the `LaTeX` Project Public License, either version 1.3c of this license or (at your option) any later version.
The latest version of this license is in `http://www.latex-project.org/lppl.txt` and version 1.3c or later is part of all distributions of LaTeX version 2005/12/01 or later.
%    \end{macrocode}
%
%</readme>
%
% \Finale

\endinput

% \PrintIndex
% makeindex -s gglo.ist -o hep-reference-implementation.gls hep-reference-implementation.glo
% makeindex -s gglo.ist -o hep-reference-implementation.ind hep-reference-implementation.idx
